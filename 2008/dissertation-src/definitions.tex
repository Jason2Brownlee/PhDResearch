%
% Definitions file for command and environment
% Jason Brownlee
%

%
% PACKAGES
%

% used for inserting images (.eps)
\usepackage{graphicx} 

% used for displaying pseudocode
% \usepackage{algorithm}
% \usepackage{algorithmic}
% even better looking algorithms
% http://andrewjpage.com/index.php?/archives/55-Algorithms-in-Latex.html
\usepackage[algochapter, ruled, linesnumbered]{algorithm2e}

% setting the text spacing in a user friendly way
% information: http://mit.edu/answers/latex/formatting/latex_spacing.html
\usepackage{setspace} 

% For table formats
% Defines a tabularx environment that is similar to tabular  but it modifies the column widths, rather than the inter-column space, to achieve the desired table width.
\usepackage{tabularx}

% doing landscape things
% http://www.hep.man.ac.uk/u/jenny/jcwdocs/latex/latex.html
\usepackage{lscape}

% for doing sub figures (graphics guide)
\usepackage{subfig}

% so i can define the margins explicitly
% http://www.biochem.ucl.ac.uk/~james/latex/thesis_master_file.html
\usepackage{geometry}

% capture URLS
\usepackage{url}

% comments
\usepackage{verbatim} 

% nice looking lines in tables
\usepackage{booktabs}

% improved positioning of float
\usepackage{float}

%
% NEW COMMMANDS
%

% new macro for starting a new page and changing the style to empty
% \newpage == ends the current page. 
% \thispagestyle == works in the same manner as the \pagestyle, except that it changes the style for the current page only. 
% empty == Produces empty heads and feet - no page numbers
\newcommand{\blanknonumber}{\newpage\thispagestyle{empty}}

% Short cut for naive and Naive:
% To use - just use the commands \naive and \Naive to get naive displayed correctly
% to get the space after the command use \naive\ and \Naive\
% \def\naive{na\"{\i}ve} 
% \def\Naive{Na\"{\i}ve} 
\newcommand{\naive}{na\"{\i}ve}
\newcommand{\Naive}{Na\"{\i}ve}

% naivete
% \newcommand{\naivete}{naivet\'{\e}}

% for defining definitions, example:
% FROM: http://www.kronto.org/thesis/tips/unnumbered-theorem.html
%\begin{definition}\label{def:data-parallelism}
%  Data-parallel programming is a form of parallel programming in which
%  the programmer specifies the distribution of data over the processors.
%  It is left to the compiler to choose and implement the distribution of
%  the actual computations over the processors.
%\end{definition}
% \newtheorem*{definition}{Definition}
% \newtheorem*{principle}{Principle}